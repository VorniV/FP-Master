\section{Theorie}
\label{sec:Theorie}

\subsection{Zielsetzung}
Ziel des Versuches ist es, die Lebensdauer kosmischer Myonen zu bestimmen. 
Zudem soll die dafür verwendete Schaltung genauer untersucht werden.


\subsection{Entstehung und Zerfall kosmischer Myonen}
Treffen hoch energetische kosmische Teilchen auf die Erdatmosphäre, so entstehen sogenannte Luftschauer, in denen wiederum Myonen entstehen.
In diesen Luftschauern entstehen durch unterschiedliche Prozesse Pionen, welche eine kurze Lebensdauer besitzen und anschließend zu Myonen zerfallen

\begin{align*}
    \pi^+ \rightarrow \mu^+ + \nu_\mu \\
    \pi^- \rightarrow \mu^- + \bar \nu_\mu.
\end{align*}
Die entstandenen Myonen bewegen sich mit nahezu Lichtgeschwindigkeit Richtung Erde. 
Durch die Relativistische Geschwindigkeit erreichen genug Myonen die Erdoberfläche, um sie dort mit einem Szintillator zu detektieren.
Treten die Myonen in den Szintillator ein, so wechselwirken sie mit der Szintillatormaterie.
Dabei geben sie einen Teil ihrer Energie an die mit ihnen wechselwirkenden Moleküle ab.
Dadurch werden diese Moleküle angeregt und fallen nach einiger Zeit zurück in ihren Grundzustand.
Bei dieser Relaxation wird ein Photon emittiert, welches detektiert werden kann und somit das Eintreffen eines Myons in den Szintillator.
Ein Teil der Myonen zerfällt zudem innerhalb des Szintillators in ein Elektron und zwei Neutrinos

\begin{align*}
    \mu^- \rightarrow e^- + \bar \nu_e + \nu_\mu \\
    \mu^+ \rightarrow e^+ + \nu_e + \bar \nu_\mu.
\end{align*}
Die dabei entstehenden Elektronen erzeugen ebenfalls Photonen innerhalb des Detektors, wodurch der zeitliche Abstand zwischen eintreten eines Myons und seinem Zerfall gemessen werden kann.
Dieser zeitliche Abstand entspricht der individuellen Lebensdauer des jeweiligen Myons.


\subsection{Lebensdauer der Myonen}
Der Zerfall eines Elementarteilchens wie das Myon ist ein statistischer Prozess.
Jedes Teilchen hat die selbe Wahrscheinlichkeit d$W$ zu zerfallen.
Diese Wahrscheinlichkeit ist proportional zur Zeit d$t$.
Daraus folgt: 

\begin{equation}
    \text{d}W = \lambda \text{d}t.
\end{equation}
Für den Zerfall von N Teilchen gilt somit

\begin{equation}
    \text{d}N = - N \text{d}W = - \lambda N \text{d}t.
\end{equation}
Dabei ist d$N$ die Anzahl der Teilchen die im, Zeitraum d$t$ zerfallen, wenn N Teilchen beobachtet werden.
Durch integrieren der Gleichung folgt das Zerfallsgesetz:

\begin{equation}
    N(t) = N_0 \cdot \text{exp}(-\lambda t).
\end{equation}
Dabei ist $\lambda$ die Zerfallskonstante, t die Zeit und $N_0$ die Anzahl der betrachteten Teilchen.
Die Lebensdauer $\tau$ entspricht nun der Zeit $t$, nach der die noch nicht zerfallenen Teilchen $N(t) = \frac{N_0}{e}$ entsprechen.
Die Lebensdauer kann durch $\tau = \frac{1}{\lambda}$ berechnet werden.
