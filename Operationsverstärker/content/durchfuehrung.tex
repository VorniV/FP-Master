
\section{Durchführung}
\label{sec:Durchführung}
Die in Kapitel \ref{sec:Theorie} dargestellten Schaltungen, werden nach vorbild ihrer jeweiligen Abbildungen aufgebaut und getestet.

\begin{itemize}
    \item Bei dem invertierten Linearverstärker wird eine Sinusspannung eingegeben und anschließend für verschiedene Frequenzen die Amplitude, so wie die Phasendifferenz zwischen Eingangs- und Ausgangsspannung gemessen.
        Dies wird für drei verschiedene Verhältnisse zwischen $R_1$ und $R_2$ durchgeführt.

    \item Beim Integrator, so wie Differentiator werden ebenfalls die Amplituden der Ausgangsspannung bei verschiedenen Frequenzen einer eingehenden Sinusspanung untersucht.
        Zusätzlich werden über einen Signalgenerator Dreiecks- und Rechtecksspannungen als eingangsspannung genutzt und das Verhalten der Ausgangsspannung mit hilfe eines Oszilloskops untersucht.

    \item Beim Schmitt-Trigger wird der Schwellenwert gesucht. Dafür wird die Eingangsspannung so lange erhöht, bis die Ausgansspannung ihr Vorzeichen wechselt.
        Zudem wird der Schwellenwert mit Hilfe einer Dreiecksspannung ermittelt. Dafür wird eine Dreiecksspannung als eingangsspannung angelegt, die deutlich größer als der Schwellenwert ist.
    	Durch ein Oszilloskop kann dann der Schwellenwert abgelesen werden.

    \item Beim Signalgenerator werden sowohl die Ausgangsspannungdes Schmitt-Triggers, so wie die des Integrators, auf einem Oszilloskop abgebildet und verglichen.

    \item Bei der gedämpften harmonischen Schwingung werden die Eingangs- und Ausgangsspannung auf dem Oszilloskop dargestellt und untersucht.

\end{itemize}
