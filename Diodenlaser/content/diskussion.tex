\newpage
\section{Diskussion}
\label{sec:Diskussion}
Bei allen durchgeführten Messungen stimmen die Ergebnisse mit der Theorie überein.
Die Lasergranulation ist eindeutig erkennbar und somit der Schwellstrom, ab dem von LED-Betrieb auf Laser-Betrieb gewechselt wird, bestimmbar.
Die Rubidiumfluoriszenz ist auf dem Bild erkennbar, auch wenn nur eine leichte Fluoriszenz auftritt.
Die Absorptionslinien hingegen sind sehr deutlich zu erkennen. 
Durch den Verglech zwischen unterdrücktem Untergrund und nicht unterdrücktem Untergrund ist gut zu sehen , dass die Unterdrückung mithilfe des 50/50-Teilers erfolgreich ist.
Zudem ist an der geraden Untergrundkurve (bei unterdrücktem Untergrund) zu erkennen, dass es keine Modensprünge gibt.
Insgesammt kann der Versuch somit als gelungen betrachtet werden.